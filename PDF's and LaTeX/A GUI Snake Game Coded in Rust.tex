\documentclass{article}
\usepackage{blindtext}
\usepackage[paperheight=107in,paperwidth=10in,margin=1in]{geometry}
\usepackage[T1]{fontenc}
\usepackage[utf8]{inputenc}
\usepackage{lmodern}
\pagenumbering{gobble}

\usepackage{minted}
\usepackage{titlesec}
\titleformat{\section}{\normalfont\Large\bfseries}{}{0pt}{}

\begin{document}


\title{A GUI Snake Game Coded in Rust}
\author{Brighton Sikarskie}
\date{2021}
\maketitle

\section{LICENSE}
\begin{minted}{md}
MIT License

Copyright (c) 2021 Brighton Sikarskie

Permission is hereby granted, free of charge, to any person obtaining a copy
of this software and associated documentation files (the "Software"), to deal
in the Software without restriction, including without limitation the rights
to use, copy, modify, merge, publish, distribute, sublicense, and/or sell
copies of the Software, and to permit persons to whom the Software is
furnished to do so, subject to the following conditions:

The above copyright notice and this permission notice shall be included in all
copies or substantial portions of the Software.

THE SOFTWARE IS PROVIDED "AS IS", WITHOUT WARRANTY OF ANY KIND, EXPRESS OR
IMPLIED, INCLUDING BUT NOT LIMITED TO THE WARRANTIES OF MERCHANTABILITY,
FITNESS FOR A PARTICULAR PURPOSE AND NONINFRINGEMENT. IN NO EVENT SHALL THE
AUTHORS OR COPYRIGHT HOLDERS BE LIABLE FOR ANY CLAIM, DAMAGES OR OTHER
LIABILITY, WHETHER IN AN ACTION OF CONTRACT, TORT OR OTHERWISE, ARISING FROM,
OUT OF OR IN CONNECTION WITH THE SOFTWARE OR THE USE OR OTHER DEALINGS IN THE
SOFTWARE.
\end{minted}

\section{README.md}
\begin{minted}{md}
# Snake

## Game

<p align="left">
<img alt="snake beginning" src="assets/snake beginning.png"/>
<img alt="snake during" src="assets/snake during.png"/>
<img alt="snake final" src="assets/snake final.png"/>
<img alt="snake gameplay" src="assets/snake gameplay.gif"/>
</p>

## Information for College Board
The code that I am using that is not a part of the standard library are:

rand:
https://crates.io/crates/rand</br>
piston_window:
https://crates.io/crates/piston_window</br>
piston2d-opengl_graphics:
https://crates.io/crates/piston2d-opengl_graphics</br>

**** NOTE ****</br>
Rust has a really small standard library so it is common to 'import' others code</br>
for more information about this read this:</br>
https://users.rust-lang.org/t/rust-should-have-a-big-standard-library-and-heres-why/37449</br>
it talks about making rust have a larger standard library and the creaters of the</br>
language shut this down listing the reasons for not having a large library.

Also refer to this to learn some more about cargo (the package manager for rust)</br>
https://doc.rust-lang.org/stable/book/ch01-03-hello-cargo.html

Cargo is a convention and is standard even though I am taking code from a third party source</br>
it is standard.

The only asset that I used in this code is a font:

FiraSans-Regular.ttf:
https://www.ffonts.net/Fira-Sans-Regular.font

## Setup and run
Install rust:</br>
https://www.rust-lang.org/tools/install</br>
be in the root directory and run:
<code>cargo run</code>
\end{minted}

\section{Cargo.toml}
\begin{minted}{rust}
[package]
name = "snake"
version = "0.1.0"
authors = ["Brighton Sikarskie <brighton.sikarskie@tuta.io>"]
edition = "2018"
description = "A simple GUI rust snake game"
readme = "README.md"
homepage = "https://github.com/bsikar/snake"
repository = "https://github.com/bsikar/snake/"
license = "MIT"
keywords = ["gamedev", "graphics", "game", "gui"]
categories = ["games"]

[dependencies]
rand = "0.8.0"
piston_window = "0.119.0"
piston2d-opengl_graphics = "0.78.0"
\end{minted}

\section{main.rs}
\begin{minted}{rust}
mod draw;
mod food;
mod game;
mod snake;

use draw::BLOCK_SIZE;
use food::Food;
use game::{Color, Game};
use piston_window::*;
use snake::Snake;

fn main() {
    let mut window: PistonWindow = WindowSettings::new("Snake Game", [400, 400])
        .exit_on_esc(true)
        .build()
        .unwrap_or_else(|e| panic!("Failed to build PistonWindow: {}", e));

    let mut game = Game::new(
        Snake::new(
            (window.size().width / (BLOCK_SIZE * 2.0)) as i32,
            (window.size().height / (BLOCK_SIZE * 2.0)) as i32,
        ),
        Food::new(window.size()),
        window.size(),
    );

    while let Some(e) = window.next() {
        if let Some(args) = e.render_args() {
            game.draw_instructions(args);
        }

        if let Some(Button::Keyboard(_)) = e.press_args() {
            break;
        }
    }

    let mut x = Key::Q;

    while let Some(e) = window.next() {
        if let Some(Button::Keyboard(k)) = e.press_args() {
            if k != x {
                x = k;
            }
        }

        window.draw_2d(&e, |c, g, _| {
            clear(Color::BACKGROUND, g);
            game.draw(&c, g);
        });

        e.update(|args| {
            game.update(window.size(), args, x);
        });

        if game.over() {
            if let Some(args) = e.render_args() {
                game.draw_game_over(args);
            }
        }
    }
}
\end{minted}

\section{game.rs}
\begin{minted}{rust}
use crate::food::Food;
use crate::snake::*;
use opengl_graphics::{GlGraphics, GlyphCache};
use piston_window::*;

#[non_exhaustive]
pub struct Color;

impl Color {
    pub const BACKGROUND: [f32; 4] = [0.3, 0.4, 0.2, 1.0];
    pub const SNAKE_BODY: [f32; 4] = [0.0, 0.0, 1.0, 1.0];
    pub const SNAKE_HEAD: [f32; 4] = [0.3, 0.6, 1.2, 1.0];
    pub const FOOD: [f32; 4] = [1.0, 0.6, 0.2, 1.0];
    pub const TEXT: [f32; 4] = [1.0, 0.99, 0.22, 1.0];
}

#[derive(Copy, Clone, Debug, PartialEq)]
pub struct Position {
    pub x: i32,
    pub y: i32,
}

#[derive(Debug, PartialEq)]
pub struct Game {
    pub snake: Snake,
    pub food: Food,
    pub window_size: Size,
}

impl Game {
    // make a new game
    pub fn new(snake: Snake, food: Food, size: Size) -> Game {
        Game {
            snake,
            food,
            window_size: size,
        }
    }
    // call a function to draw the instructions screen
    pub fn draw_instructions(&self, args: RenderArgs) {
        let mut gl = GlGraphics::new(OpenGL::V3_2);
        let font = include_bytes!("../assets/FiraSans-Regular.ttf");
        let mut glyphs = GlyphCache::from_bytes(font, (), TextureSettings::new()).unwrap();

        gl.draw(args.viewport(), |c, g| {
            clear(Color::BACKGROUND, g);
            text(
                Color::TEXT,
                (self.window_size.width / 25.0) as u32,
                "WASD or Arrow Keys: Move",
                &mut glyphs,
                c.transform
                    .trans(self.window_size.width / 4.0, self.window_size.height / 2.0),
                g,
            )
        })
        .expect("Failed to make end screen");

        gl.draw(args.viewport(), |c, g| {
            text(
                Color::TEXT,
                (self.window_size.width / 25.0) as u32,
                "Q: Pause",
                &mut glyphs,
                c.transform
                    .trans(self.window_size.width / 4.0, self.window_size.height / 2.35),
                g,
            )
        })
        .expect("Failed to make end screen");

        gl.draw(args.viewport(), |c, g| {
            text(
                Color::TEXT,
                (self.window_size.width / 25.0) as u32,
                "Esc: Quit",
                &mut glyphs,
                c.transform
                    .trans(self.window_size.width / 4.0, self.window_size.height / 1.75),
                g,
            )
        })
        .expect("Failed to make end screen");
    }

    // update the game by calling functions to move the snake
    // have the snake eat food and spawn new food
    pub fn update(&mut self, size: Size, args: &UpdateArgs, key: Key) {
        self.snake.update(size, args.dt, self.key_direction(key));
        self.window_size = size;
        if self.snake.position == self.food.position {
            self.snake.eat();
            self.food.spawn(size, &self.snake);
        }
    }

    // change the direction the snake is moving based on the players
    // keyboard input
    fn key_direction(&self, key: Key) -> Direction {
        return {
            match key {
                Key::Right | Key::D => Direction::Right,
                Key::Left | Key::A => Direction::Left,
                Key::Down | Key::S => Direction::Down,
                Key::Up | Key::W => Direction::Up,
                Key::Q => Direction::Still,
                _ => self.snake.direction,
            }
        };
    }

    // call functions to draw the snake and the food
    pub fn draw(&mut self, c: &Context, g: &mut G2d) {
        self.snake.draw(c, g);
        self.food.draw(c, g);
    }

    // return is the game is over or not (if the snake is dead)
    pub fn over(&self) -> bool {
        !self.snake.is_alive()
    }

    // draw the game over screen and show the final length of the snake
    pub fn draw_game_over(&self, args: RenderArgs) {
        let mut gl = GlGraphics::new(OpenGL::V3_2);
        let font = include_bytes!("../assets/FiraSans-Regular.ttf");
        let mut glyphs = GlyphCache::from_bytes(font, (), TextureSettings::new()).unwrap();

        gl.draw(args.viewport(), |c, g| {
            clear(Color::BACKGROUND, g);
            text(
                Color::TEXT,
                (self.window_size.width / 13.3) as u32,
                format!("Final Length: {}", self.snake.length).as_str(),
                &mut glyphs,
                c.transform
                    .trans(self.window_size.width / 4.0, self.window_size.height / 2.0),
                g,
            )
        })
        .expect("Failed to make end screen");
    }
}
\end{minted}

\section{snake.rs}
\begin{minted}{rust}
use crate::draw::{draw, BLOCK_SIZE};
use crate::game::{Color, Position};
use piston_window::{Context, G2d, Size};
use std::collections::VecDeque;

const SNAKE_WAIT: f64 = 0.2;

#[derive(Debug, PartialEq, Copy, Clone)]
pub enum Direction {
    Left,
    Right,
    Up,
    Down,
    Still,
}

#[derive(Clone, Debug, PartialEq)]
pub struct Snake {
    pub position: Position,
    pub length: u32,
    pub direction: Direction,
    pub tail: VecDeque<Position>,
    is_alive: bool,
    wait: f64,
}

impl Snake {
    // make a new snake
    pub fn new(x: i32, y: i32) -> Snake {
        Snake {
            position: Position { x, y },
            length: 1,
            direction: Direction::Still,
            tail: vec![].into_iter().collect(),
            is_alive: true,
            wait: 0.0,
        }
    }

    // check if the position of the snake is out of bounds
    fn is_valid(&self, size: Size) -> bool {
        let x = self.position.x;
        let y = self.position.y;
        x >= 0
            && y >= 0
            && x <= (size.width / BLOCK_SIZE) as i32
            && y <= (size.height / BLOCK_SIZE) as i32
    }

    // move the snake in the direction it is facing
    pub fn mv(&mut self, size: Size, direction: Direction) {
        self.wait = 0.0;
        if !self.is_valid(size) {
            self.is_alive = false;
            return;
        }

        match self.direction {
            Direction::Left => {
                if direction != Direction::Right {
                    self.direction = direction;
                }
            }
            Direction::Right => {
                if direction != Direction::Left {
                    self.direction = direction;
                }
            }
            Direction::Up => {
                if direction != Direction::Down {
                    self.direction = direction;
                }
            }
            Direction::Down => {
                if direction != Direction::Up {
                    self.direction = direction;
                }
            }
            Direction::Still => self.direction = direction,
        }

        // Note: I am using 2 match cases here for visibilty (I could have put this in the one up above).
        match self.direction {
            Direction::Left => {
                if self.overlap_tail(self.position.x - 1, self.position.y) {
                    self.is_alive = false;
                    return;
                }
                self.position.x -= 1;
                self.tail.pop_back();
                self.tail.push_front(self.position);
            }
            Direction::Right => {
                if self.overlap_tail(self.position.x + 1, self.position.y) {
                    self.is_alive = false;
                    return;
                }
                self.position.x += 1;
                self.tail.pop_back();
                self.tail.push_front(self.position);
            }
            Direction::Up => {
                if self.overlap_tail(self.position.x, self.position.y - 1) {
                    self.is_alive = false;
                    return;
                }
                self.position.y -= 1;
                self.tail.pop_back();
                self.tail.push_front(self.position);
            }
            Direction::Down => {
                if self.overlap_tail(self.position.x, self.position.y + 1) {
                    self.is_alive = false;
                    return;
                }
                self.position.y += 1;
                self.tail.pop_back();
                self.tail.push_front(self.position);
            }
            Direction::Still => {}
        }
    }

    // return if the snake is over laping its tail
    fn overlap_tail(&self, x: i32, y: i32) -> bool {
        self.tail.contains(&Position { x, y })
    }

    // have the snake eat food updating the snakes length
    pub fn eat(&mut self) {
        match self.direction {
            Direction::Left => {
                self.tail.push_back(Position {
                    x: self.position.x + 1,
                    y: self.position.y,
                });
            }
            Direction::Right => {
                self.tail.push_back(Position {
                    x: self.position.x - 1,
                    y: self.position.y,
                });
            }
            Direction::Up => {
                self.tail.push_back(Position {
                    x: self.position.x,
                    y: self.position.y + 1,
                });
            }
            Direction::Down => {
                self.tail.push_back(Position {
                    x: self.position.x,
                    y: self.position.y - 1,
                });
            }
            Direction::Still => {}
        }
        self.length += 1;
    }

    // draw the snake
    /*
    pub fn draw(&self, c: &Context, g: &mut G2d) {
        draw(
            Color::SNAKE_HEAD,
            self.position.x as u32,
            self.position.y as u32,
            1,
            1,
            c,
            g,
        );
        self.tail
            .iter()
            .skip(1)
            .for_each(|seg| draw(Color::SNAKE_BODY, seg.x as u32, seg.y as u32, 1, 1, c, g));
    }
    */
    // NOTE I am rewriting this to remove confusion by the .for_each()
    // I would have used the code above, but I do not want to get points
    // remove by someone who is confused if I was accessing multiple
    // elements in the list (VecDeque)
    pub fn draw(&self, c: &Context, g: &mut G2d) {
        draw(
            Color::SNAKE_HEAD,
            self.position.x as u32,
            self.position.y as u32,
            1,
            1,
            c,
            g,
        );
        for seg in self.tail.iter().skip(1) {
            draw(Color::SNAKE_BODY, seg.x as u32, seg.y as u32, 1, 1, c, g);
        }
    }

    // return if the snake is alive
    pub fn is_alive(&self) -> bool {
        self.is_alive
    }

    // move the snake after a set amount of 'wait time' so the snake isnt too fast
    pub fn update(&mut self, size: Size, dt: f64, direction: Direction) {
        self.wait += dt;
        if self.wait > SNAKE_WAIT {
            self.mv(size, direction);
        }
    }
}
\end{minted}

\section{food.rs}
\begin{minted}{rust}
use crate::draw::{draw, BLOCK_SIZE};
use crate::game::{Color, Position};
use crate::snake::Snake;
use piston_window::{Context, G2d, Size};
use rand::{thread_rng, Rng};

#[derive(Debug, PartialEq)]
pub struct Food {
    pub position: Position,
}

impl Food {
    // make a new food
    pub fn new(size: Size) -> Food {
        Food {
            position: Position {
                x: thread_rng().gen_range(0..=(size.width / (BLOCK_SIZE * 2.0)) as i32),
                y: thread_rng().gen_range(0..=(size.height / (BLOCK_SIZE * 2.0)) as i32),
            },
        }
    }

    // spawn the food on the screen in a valid location
    /*
    pub fn spawn(&mut self, size: Size, snake: &Snake) {
        while snake.tail.contains(&self.position) {
            self.position = Position {
                x: thread_rng().gen_range(0..=(size.width / (BLOCK_SIZE * 2.0)) as i32),
                y: thread_rng().gen_range(0..=(size.height / (BLOCK_SIZE * 2.0)) as i32),
            };
        }
    }
    */
    // NOTE I am rewriting this to fit the college board requirements
    // I would have used the code above if it fit the requirements
    pub fn spawn(&mut self, size: Size, snake: &Snake) {
        loop {
            let mut thing: bool = true;
            for i in snake.tail.iter() {
                if i == &self.position {
                    thing = false;
                    break;
                } else {
                    thing = true;
                }
            }

            if thing {
                break;
            } else {
                self.position = Position {
                    x: thread_rng().gen_range(0..=(size.width / (BLOCK_SIZE * 2.0)) as i32),
                    y: thread_rng().gen_range(0..=(size.height / (BLOCK_SIZE * 2.0)) as i32),
                };
            }
        }
    }

    // draw the food on screen
    pub fn draw(&self, c: &Context, g: &mut G2d) {
        draw(
            Color::FOOD,
            self.position.x as u32,
            self.position.y as u32,
            1,
            1,
            c,
            g,
        );
    }
}
\end{minted}

\section{draw.rs}
\begin{minted}{rust}
use piston_window::types::Color;
use piston_window::{rectangle, Context, G2d};

pub const BLOCK_SIZE: f64 = 25.0;

// return input pixel to the size of a block
// (the head or 1 body segment of the snake)
pub fn to_block_size(size: u32) -> f64 {
    f64::from(size) * BLOCK_SIZE
}

// draw a rectangle on the screen with the parameters inputted
pub fn draw(color: Color, x: u32, y: u32, width: u32, height: u32, c: &Context, g: &mut G2d) {
    rectangle(
        color,
        [
            to_block_size(x),
            to_block_size(y),
            to_block_size(width),
            to_block_size(height),
        ],
        c.transform,
        g,
    );
}
\end{minted}

\end{document}
